\documentclass[aspectratio=43]{beamer}

% Import packages
\usepackage{luatexja}
\usepackage[symbol]{footmisc}

% Hyperref setting
\usepackage{hyperref}
\hypersetup{
    colorlinks=false,
}

% Margin setting
\setbeamersize{text margin left=5mm,text margin right=5mm}

% Beamer template settings
\usetheme{Montpellier}
\usecolortheme[RGB={140,0,0}]{structure}
\setbeamertemplate{navigation symbols}{}

% Beamer color box settings
\setbeamercolor{block title}{bg=black, fg=white}
\setbeamercolor{block body}{bg=black!10, fg=black}
\setbeamercolor{block title alerted}{bg=red!70!black, fg=white}
\setbeamercolor{block body alerted}{bg=red!10, fg=black}

% Font setting
\renewcommand{\familydefault}{\rmdefault}
\usefonttheme{professionalfonts}

% Caption and footnote settings
\renewcommand{\figurename}{図}
\renewcommand{\tablename}{表}
\renewcommand{\thefootnote}{\ \fnsymbol{footnote}}
\renewcommand{\thempfootnote}{\ \fnsymbol{mpfootnote}}

% Biber settings
\usepackage[backend=biber, style=numeric, sorting=none, isbn=false, url=false, doi=false,eprint=false]{biblatex}
\addbibresource{bib.bib}
\setbeamertemplate{bibliography item}[text]

% itemize & enumirate settings
\setbeamertemplate{emumerate item}[default]
\setbeamertemplate{itemize item}[triangle]

% Other settings
\makeatletter
\@addtoreset{footnote}{page}
\makeatother

% TikZ settings
\input{TikZsetting.tset}

\title{ワンタイムパスワードによるIEEE802.1X認証}
\subtitle{(修士論文にしたい)}
\author[K.MIZOGUCHI]{溝口 洸熙\thanks{高知工科大学 情報学群 情報セキュリティシステム研究室}}
\date{\today}

% Footer setting
\setbeamertemplate{footline}{
    \leavevmode
    \hbox{\begin{beamercolorbox}[wd=.95\paperwidth, ht=1ex, right,dp=1.5ex, rightskip=0mm]{structure}
            \insertframenumber{} / \inserttotalframenumber
        \end{beamercolorbox}}
}

% Table of Contents and Frame title settings
\newcommand{\stoc}{1-2}
\newcommand{\ftoc}{3}
\newcommand{\ft}{\thesection.\thesubsection\ \subsecname}
\newcommand{\fft}{\thesection.\ \secname}
\newcommand{\toc}{
    \begin{frame}[t]{\fft}
        \begin{columns}[t]
            \begin{column}{.48\textwidth}
                \tableofcontents[sections=\stoc, currentsection,sectionstyle=show/shaded, subsectionstyle=show/show/shaded]
            \end{column}
            \begin{column}{.48\textwidth}
                \tableofcontents[sections=\ftoc, currentsection,sectionstyle=show/shaded, subsectionstyle=show/show/shaded]
            \end{column}
        \end{columns}
    \end{frame}
}
\setbeamertemplate{section in toc}{
    {\color{structure}\inserttocsectionnumber}.\
    \inserttocsection\vspace{.2em}
}
\setbeamertemplate{subsection in toc}{
    \hspace{1em}{\color{structure}\rule[.3ex]{3pt}{3pt}}\
    \inserttocsubsection\par\vspace{.2em}
}


\begin{document}
\maketitle

% Table of Contents ---
\begin{frame}[t]{Table of Contents}
    \begin{columns}[t]
        \begin{column}{.48\textwidth}
            \tableofcontents[sections=\stoc]
        \end{column}
        \begin{column}{.48\textwidth}
            \tableofcontents[sections=\ftoc]
        \end{column}
    \end{columns}
\end{frame}


\section{用語解説}
\begin{frame}[c]{\fft}
    \begin{block}{IEEE802.1X}
        `` 認められた機器のみがネットワークにアクセスできるように認証する仕組み''\footfullcite{マスタリングTCPIP}
    \end{block}
    \vfill
    \begin{block}{EAP}
        EAP(Extensible Authentication Protocol:直訳すると「拡張認証プロトコル」)とは,
        RFC3748及びRFC5247で規定されている,認証プロトコル.
        これに準拠することで汎用性が高まる.(後程)
    \end{block}
\end{frame}
\begin{frame}[c]{\fft}
    \begin{block}{AAA\footfullcite{RADIUS}}
        \begin{itemize}
            \item {\large A}uthentication(認証)\\
                  相手は誰なのか?ネットワーク接続を許可するか?
            \item {\large A}uthorization(承認)\\
                  どのようなサービスを許可するか?\\(ネットーワークアクセス制限など)
            \item {\large A}ccounting(課金)\\
                  情報の収集.接続時間,送受信したデータ量など.
        \end{itemize}
    \end{block}
\end{frame}
\begin{frame}[c]{\fft}
    \begin{block}{RADIUS\footfullcite{RADIUS}}
        \begin{itemize}
            \item 相手が本当に名乗っている通りのユーザなのか確認.
            \item ユーザが何にアクセス出来るのかの判断.
            \item そういったことを全て知らせる処理をする.
        \end{itemize}
        これらの全てに対応したプロトコル.\par
        AAAはRADIUSプロトコルの動作を保証するための汎用的なバックグラウンド.
    \end{block}
\end{frame}
\section{ワンタイムパスワード}
\toc
\begin{frame}[c]{\fft}
    \begin{block}{ワンタイムパスワード}
        ``ネットワーク上の認証において,毎回変化する1度限りのパスワード。通信経路上でパスワードを盗まれた場合でも,次回のアクセスでは無効となる。実装方法には,チャレンジ アンド レスポンス方式などがある。使い捨てパスワード。OTP。''\footfullcite{スーパー大辞林}
    \end{block}
    \vfill
    今回用いるのは,\textbf{超軽量}かつ\textbf{高速}のワンタイムパスワード認証方式,\textbf{\color{red} SAS-L(1)}\footnote{清水明宏教授考案}を採用する.
\end{frame}
\subsection{SAS-L認証フロー(概要)}
\begin{frame}{\ft}
    \begin{figure}
    \centering
    \begin{tikzpicture}[remember picture]
        %  -- TILE --
        \node[terminal](cl){Supplicant};
        \node[terminal, right=3cm of cl](sv){Server};
        \node[key={cl}{0.5}](ck1){AuthCode1};
        \node[key={sv}{0.5}](sk1){AuthCode1};
        \node[key={ck1}{3.5}](ck2){AuthCode2};
        \node[key={sk1}{3.5}](sk2){AuthCode2};
        \node[decision, below=.5cm of sk1](decision){AuthCode1 ?};
        \node[decision] at ($(ck1|-decision.east)!0.5!(ck2.north)$)(decision2){AuthCode1' ?};

        % -- ARROW --
        \draw[Stealth-Stealth,ultra thick, draw=blue!50](ck1)--(sk1)node[midway,above]{\small 初回登録};
        \begin{scope}[-Stealth, ultra thick, draw=red!50]
            \draw(ck1)|-(decision.west)node[midway, above right]{\small 認証子送信};
            \draw(decision.south)|-(decision2)node[midway, above left]{\small 認証子送信};
        \end{scope}

        \foreach \u \v in {sk1/decision,decision2.south/ck2,ck1/decision2.north,decision.south/sk2}{
                \draw[-latex](\u)--(\v);
            }
        \node[below right] at (decision.south){\scriptsize\ttfamily True};
        \node[below right] at (decision2.south){\scriptsize\ttfamily True};
    \end{tikzpicture}
\end{figure}
\begin{tikzpicture}[remember picture, overlay]
    \node[above right]at(decision.east){\ttfamily\scriptsize False};
    \draw[-latex, dashed](decision.east)-|($(decision.east)+(.5cm,-.5cm)$)node[below]{\ttfamily Failed};
    \node[above left]at(decision2.west){\ttfamily\scriptsize False};
    \draw[-latex, dashed](decision2.west)-|($(decision2.west)+(-.5cm,-.5cm)$)node[below]{\ttfamily Failed};
\end{tikzpicture}
\end{frame}
\subsection{解決すべき課題}
\begin{frame}[t]{\ft}
    \begin{alertblock}{相互認証\hfill\textbf{\fbox{\scriptsize 解決済}}}
        現行のSAS-L(1)は,
        \begin{itemize}
            \item サーバはクライアントを認証するが,
            \item クライアントはサーバを認証しない.
        \end{itemize}
        IEEE802.1X/EAP認証には様々な種類がある.
        クライアントがサーバを認証しないEAP-MD5は,\textbf{クライアントが\underline{不正な}サーバにアクセスする可能性}を考える.
        認証プロトコルが相互認証機能を有する必要がある.
    \end{alertblock}
    \begin{alertblock}{同期ずれ\hfill\textbf{\fbox{\scriptsize 解決済}}}
        OTPは,同一の認証子をクライアントで保持する必クライアントで保持する必要がある.
        何らかの障害で,保持する認証子が同一でなくなる現象(同期ずれ)を解決する必要がある.
    \end{alertblock}
\end{frame}
\section{IEEE802.1X認証}
\toc
\subsection{構成図}
\begin{frame}{\ft}
    \begin{figure}
    \centering
    \tikzset{base/.style={text width=6cm, minimum height=1cm, text centered, font=\ttfamily},
        supplicant/.style={base, draw, rounded corners, text width=2.8cm},
        switch/.style={base, draw, double, rectangle, font=\normalfont, fill=gray!10},
        server/.style={base, draw, thick, rectangle}
    }
    \begin{tikzpicture}[remember picture]
        \node[switch](switch){\ttfamily Swich supported IEEE802.1X};
        \node[server,above=1cm of switch](radius){RADIUS Server};
        \node[supplicant,below=1cm of switch.south west, anchor=north west](sc1){Supplicant};
        \node[supplicant,below=1cm of switch.south east, anchor=north east](sc2){Supplicant};

        \begin{scope}[red, thick, latex-latex]
            \draw(sc1.north)--(sc1.north |- switch.south)node[midway](eapol1){};
            \draw(sc2.north)--(sc2.north |- switch.south)node[midway](eapol2){};
            \node at ($(eapol1)!0.5!(eapol2)$){\ttfamily EAP over LAN};
        \end{scope}
        \begin{scope}[blue, thick, latex-latex]
            \draw(switch.north)--(radius.south)node[midway, right]{\ttfamily EAP over RADIUS};
        \end{scope}
    \end{tikzpicture}
\end{figure}
\begin{tikzpicture}[remember picture, overlay]
    \node[below right=1cm of switch.east, fill=cyan!10, rounded corners, minimum height=1cm](internet){\ttfamily Internet};
    \draw[dashed, Stealth-Stealth, thick](switch.east)-|(internet);
\end{tikzpicture}
\end{frame}
\subsection{プロトコルレイヤ}
\begin{frame}{\ft}
    \begin{figure}
    \centering
    \tikzset{box/.style n args={2}{text width=2cm, text centered, font=\ttfamily\scriptsize, minimum height=#1,fill=#2, align=center}}
    \begin{tikzpicture}
        \node[box={3cm}{green!30}](s1){IEEE802.3\\IEEE802.11};
        \node[box={1cm}{green!10},above=0cm of s1](s2){EAPOL};
        \node[box={1cm}{red!30},above=0cm of s2](s3){EAP};
        \node[box={1cm}{red!10},above=0cm of s3](s4){Application layer};

        \node[box={3cm}{green!30},right=1cm of s1](sw1){IEEE802.3\\IEEE802.11};
        \node[box={1cm}{green!10},above=0cm of sw1](sw2){EAPOL};

        \node[box={1cm}{blue!40},right=0cm of sw1.south east, anchor=south west](r1){Ethernet};
        \node[box={1cm}{blue!30},above=-0.4pt of r1](r2){IP};
        \node[box={1cm}{blue!20},above=-0.4pt of r2](r3){UDP};
        \node[box={1cm}{blue!10},above=0cm of r3](r4){RADIUS};

        \node[box={1cm}{blue!40},right=1cm of r1](rs1){Ethernet};
        \node[box={1cm}{blue!30},above=-0.4pt of rs1](rs2){IP};
        \node[box={1cm}{blue!20},above=-0.4pt of rs2](rs3){UDP};
        \node[box={1cm}{blue!10},above=0cm of rs3](rs4){RADIUS};
        \node[box={1cm}{red!30},above=0cm of rs4](rs5){EAP};
        \node[box={1cm}{red!10},above=0cm of rs5](rs6){Application layer};

        \begin{scope}[very thick, latex-latex]
            \draw(s2)--(sw2);
            \draw(s3)--(rs5);
            \draw(s4)--(rs6);
            \draw(r4)--(rs4);
        \end{scope}
        \node[below]at(s1.south){\ttfamily\scriptsize Supplicant};
        \node[below]at($(sw1.south)!0.5!(r1.south)$){\ttfamily\scriptsize Swich supported IEEE802.1X};
        \node[below]at(rs1.south){\ttfamily\scriptsize RADIUS Server};
    \end{tikzpicture}
\end{figure}
\end{frame}
\subsection{認証フロー}
\begin{frame}{\ft}
    \begin{figure}
    \centering
    \begin{tikzpicture}
        \node(sp){\ttfamily Supplicant};
        \node[right=6cm of sp](sv){\ttfamily Server};
        \draw[ultra thick](sp)--($(sp.south)+(0,-6cm)$);
        \draw[ultra thick](sv)--($(sv.south)+(0,-6cm)$);

        \begin{scope}[thick]
            \draw[latex-latex]($(sp.south)+(0,-.5cm)$)--($(sv.south)+(0,-.5cm)$)node[midway,fill=white]{Association};
            \draw[draw=blue,-latex]($(sp.south)+(0,-1cm)$)--($(sv.south)+(0,-1cm)$)node[midway,fill=white]{\ttfamily EAPOL Start};
            \draw[draw=red,-latex]($(sv.south)+(0,-1.5cm)$)--($(sp.south)+(0,-1.5cm)$)node[midway,fill=white]{\ttfamily EAP-Request/Identity};
            \draw[draw=blue,-latex]($(sp.south)+(0,-2cm)$)--($(sv.south)+(0,-2cm)$)node[midway,fill=white]{\ttfamily EAP-Response/Identity};
            \draw[draw=red,-latex]($(sv.south)+(0,-2.5cm)$)--($(sp.south)+(0,-2.5cm)$)node[midway,fill=white]{\ttfamily EAP-Request/Auth};
            \draw[draw=blue,-latex]($(sp.south)+(0,-3.5cm)$)--($(sv.south)+(0,-3.5cm)$)node[midway,above]{\ttfamily EAP-Response/Auth (\(\alpha, \beta\))};
            \draw[draw=red,-latex]($(sv.south)+(0,-4.3cm)$)--($(sp.south)+(0,-4.3cm)$)node[midway,above]{\ttfamily EAP-Request/Auth (\(\gamma\))};
            \draw[draw=blue,-latex]($(sp.south)+(0,-5.1cm)$)--($(sv.south)+(0,-5.1cm)$)node[midway,above]{\ttfamily EAP-Response/Auth (\(\gamma\)\quad isCorrect?)};
            \draw[draw=red,-latex]($(sv.south)+(0,-5.9cm)$)--($(sp.south)+(0,-5.9cm)$)node[midway,above]{\ttfamily EAP-Success};
        \end{scope}
    \end{tikzpicture}
\end{figure}
\end{frame}
\section{Reference}
\begin{frame}[allowframebreaks]{Reference}
    \printbibliography
\end{frame}
\end{document}