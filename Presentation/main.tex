\documentclass[aspectratio=43]{beamer}

% Import packages
\usepackage{luatexja}
\usepackage[symbol]{footmisc}

% Hyperref setting
\usepackage{hyperref}
\hypersetup{
    colorlinks=false,
}

% Margin setting
\setbeamersize{text margin left=5mm,text margin right=5mm}

% Beamer template settings
\usetheme{Montpellier}
\usecolortheme[RGB={140,0,0}]{structure}
\setbeamertemplate{navigation symbols}{}

% Beamer color box settings
\setbeamercolor{block title}{bg=black, fg=white}
\setbeamercolor{block body}{bg=black!10, fg=black}
\setbeamercolor{block title alerted}{bg=red!70!black, fg=white}
\setbeamercolor{block body alerted}{bg=red!10, fg=black}

% Font setting
\renewcommand{\familydefault}{\rmdefault}
\usefonttheme{professionalfonts}

% Caption and footnote settings
\renewcommand{\figurename}{図}
\renewcommand{\tablename}{表}
\renewcommand{\thefootnote}{\ \fnsymbol{footnote}}
\renewcommand{\thempfootnote}{\ \fnsymbol{mpfootnote}}

% Biber settings
\usepackage[backend=biber, style=numeric, sorting=none, isbn=false, url=false, doi=false,eprint=false]{biblatex}
\addbibresource{bib.bib}
\setbeamertemplate{bibliography item}[text]

% itemize & enumirate settings
\setbeamertemplate{emumerate item}[default]
\setbeamertemplate{itemize item}[triangle]

% Other settings
\makeatletter
\@addtoreset{footnote}{page}
\makeatother

% TikZ settings
\input{TikZsetting.tset}

\title{ワンタイムパスワードによるIEEE802.1X認証}
\subtitle{(修士論文にしたい)}
\author[K.MIZOGUCHI]{溝口 洸熙\thanks{高知工科大学 情報学群 情報セキュリティシステム研究室}}
\date{\today}

% Footer setting
\setbeamertemplate{footline}{
    \leavevmode
    \hbox{\begin{beamercolorbox}[wd=.95\paperwidth, ht=1ex, right,dp=1.5ex, rightskip=0mm]{structure}
            \insertframenumber{} / \inserttotalframenumber
        \end{beamercolorbox}}
}

% Table of Contents and Frame title settings
\newcommand{\stoc}{1}
\newcommand{\ftoc}{2}
\newcommand{\ft}{\thesection.\thesubsection\ \subsecname}
\newcommand{\fft}{\thesection.\ \secname}
\newcommand{\toc}{
    \begin{frame}[t]{\fft}
        \begin{columns}[t]
            \begin{column}{.48\textwidth}
                \tableofcontents[sections=\stoc, currentsection,sectionstyle=show/shaded, subsectionstyle=show/show/shaded]
            \end{column}
            \begin{column}{.48\textwidth}
                \tableofcontents[sections=\ftoc, currentsection,sectionstyle=show/shaded, subsectionstyle=show/show/shaded]
            \end{column}
        \end{columns}
    \end{frame}
}
\setbeamertemplate{section in toc}{
    {\color{structure}\inserttocsectionnumber}.\
    \inserttocsection\vspace{.2em}
}
\setbeamertemplate{subsection in toc}{
    \hspace{1em}{\color{structure}\rule[.3ex]{3pt}{3pt}}\
    \inserttocsubsection\par\vspace{.2em}
}


\begin{document}
\maketitle

% Table of Contents ---
\begin{frame}[t]{Table of Contents}
    \begin{columns}[t]
        \begin{column}{.48\textwidth}
            \tableofcontents[sections=\stoc]
        \end{column}
        \begin{column}{.48\textwidth}
            \tableofcontents[sections=\ftoc]
        \end{column}
    \end{columns}
\end{frame}


\section{用語解説}
\begin{frame}[c]{\fft}
    \begin{block}{IEEE802.1X}
        `` 認められた機器のみがネットワークにアクセスできるように認証する仕組み''\footfullcite{マスタリングTCPIP}
    \end{block}
    \vfill
    \begin{block}{EAP}
        EAP(Extensible Authentication Protocol:直訳すると「拡張認証プロトコル」)とは,
        RFC3748及びRFC5247で規定されている,認証プロトコル.
        これに準拠することで汎用性が高まる.(後程)
    \end{block}
\end{frame}
\begin{frame}[c]{\fft}
    \begin{block}{AAA\footfullcite{RADIUS}}
        \begin{itemize}
            \item {\large A}uthentication(認証)\\
                  相手は誰なのか?ネットワーク接続を許可するか?
            \item {\large A}uthorization(承認)\\
                  どのようなサービスを許可するか?\\(ネットーワークアクセス制限など)
            \item {\large A}ccounting(課金)\\
                  情報の収集.接続時間,送受信したデータ量など.
        \end{itemize}
    \end{block}
\end{frame}
\begin{frame}[c]{\fft}
    \begin{block}{RADIUS\footfullcite{RADIUS}}
        \begin{itemize}
            \item 相手が本当に名乗っている通りのユーザなのか確認.
            \item ユーザが何にアクセス出来るのかの判断.
            \item そういったことを全て知らせる処理をする.
        \end{itemize}
        これらの全てに対応したプロトコル.\par
        AAAはRADIUSプロトコルの動作を保証するための汎用的なバックグラウンド.
    \end{block}
\end{frame}
\section{ワンタイムパスワード}
\toc
\begin{frame}[c]{\fft}
    \begin{block}{ワンタイムパスワード}
        ``ネットワーク上の認証において,毎回変化する1度限りのパスワード。通信経路上でパスワードを盗まれた場合でも,次回のアクセスでは無効となる。実装方法には,チャレンジ アンド レスポンス方式などがある。使い捨てパスワード。OTP。''\footfullcite{スーパー大辞林}
    \end{block}
    \vfill
    今回用いるのは,\textbf{超軽量}かつ\textbf{高速}のワンタイムパスワード認証方式,\textbf{\color{red} SAS-L(1)}\footnote{清水明宏教授考案}を採用する.
\end{frame}
\subsection{SAS-L認証フロー(概要)}
\begin{frame}{\ft}
    \begin{figure}
    \centering
    \begin{tikzpicture}[remember picture]
        %  -- TILE --
        \node[terminal](cl){Supplicant};
        \node[terminal, right=3cm of cl](sv){Server};
        \node[key={cl}{0.5}](ck1){AuthCode1};
        \node[key={sv}{0.5}](sk1){AuthCode1};
        \node[key={ck1}{3.5}](ck2){AuthCode2};
        \node[key={sk1}{3.5}](sk2){AuthCode2};
        \node[decision, below=.5cm of sk1](decision){AuthCode1 ?};
        \node[decision] at ($(ck1|-decision.east)!0.5!(ck2.north)$)(decision2){AuthCode1' ?};

        % -- ARROW --
        \draw[Stealth-Stealth,ultra thick, draw=blue!50](ck1)--(sk1)node[midway,above]{\small 初回登録};
        \begin{scope}[-Stealth, ultra thick, draw=red!50]
            \draw(ck1)|-(decision.west)node[midway, above right]{\small 認証子送信};
            \draw(decision.south)|-(decision2)node[midway, above left]{\small 認証子送信};
        \end{scope}

        \foreach \u \v in {sk1/decision,decision2.south/ck2,ck1/decision2.north,decision.south/sk2}{
                \draw[-latex](\u)--(\v);
            }
        \node[below right] at (decision.south){\scriptsize\ttfamily True};
        \node[below right] at (decision2.south){\scriptsize\ttfamily True};
    \end{tikzpicture}
\end{figure}
\begin{tikzpicture}[remember picture, overlay]
    \node[above right]at(decision.east){\ttfamily\scriptsize False};
    \draw[-latex, dashed](decision.east)-|($(decision.east)+(.5cm,-.5cm)$)node[below]{\ttfamily Failed};
    \node[above left]at(decision2.west){\ttfamily\scriptsize False};
    \draw[-latex, dashed](decision2.west)-|($(decision2.west)+(-.5cm,-.5cm)$)node[below]{\ttfamily Failed};
\end{tikzpicture}
\end{frame}
\subsection{解決すべき課題}
\begin{frame}[t]{\ft}
    \begin{alertblock}{相互認証\hfill\textbf{\fbox{\scriptsize 解決済}}}
        現行のSAS-L(1)は,
        \begin{itemize}
            \item サーバはクライアントを認証するが,
            \item クライアントはサーバを認証しない.
        \end{itemize}
        IEEE802.1X/EAP認証には様々な種類がある.
        クライアントがサーバを認証しないEAP-MD5は,\textbf{クライアントが\underline{不正な}サーバにアクセスする可能性}を考える.
        認証プロトコルが相互認証機能を有する必要がある.
    \end{alertblock}
    \begin{alertblock}{同期ずれ\hfill\textbf{\fbox{\scriptsize 解決済}}}
        OTPは,同一の認証子をクライアントで保持する必クライアントで保持する必要がある.
        何らかの障害で,保持する認証子が同一でなくなる現象(同期ずれ)を解決する必要がある.
    \end{alertblock}
\end{frame}
\begin{frame}[allowframebreaks]{Reference}
    \printbibliography
\end{frame}
\end{document}