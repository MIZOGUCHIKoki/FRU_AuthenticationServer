\section{研究の目的}
\toc
\subsection{従来の課題}
\begin{frame}{\ft}
    \begin{alertblock}{}
        \begin{itemize}
            \setlength{\itemsep}{1em}
            \item \textbf{相互に電子証明書認証を行う方式}(EAP-TLS,\dots)\\\vspace{.5em}
                  電子証明書の導入コスト,運用コストがとにかくかかる.
            \item \textbf{ID,パスワードを用いて認証を行う方式}(EAP-TTLS,\dots)\\\vspace{.5em}
                  ID,パスワードが漏洩すると,認証として使い物にならない.
        \end{itemize}
    \end{alertblock}
\end{frame}
\subsection{課題解決へのアプローチ}
\begin{frame}{\ft}
    \begin{exampleblock}{}
        \begin{itemize}
            \setlength{\itemsep}{1em}
            \item \textbf{機密事項漏洩のリスクを減らす}\\
                  ID,パスワードを用いた認証を用いないことで,機密事項漏洩を防ぐ.
            \item \textbf{コスト削減}\\
                  サーバとクライアントの認証に,電子証明書を用いないことで,コスト削減を図る.
            \item \textbf{処理の高速化}\\
                  電子証明書を用いた認証ではなく,単純な認証子に対するビット演算で認証を行うことで,高速化を図る.
        \end{itemize}
    \end{exampleblock}
    ここで,ワンタイムパスワードを用いたIEEE802.1X認証,「EAP-SAS」を提案する.
    \textbf{\color{red}EAPに準拠して}認証規格を作成する.
\end{frame}