\section{ワンタイムパスワード}
\toc
\begin{frame}[c]{\fft}
    \begin{block}{ワンタイムパスワード}
        ``ネットワーク上の認証において,毎回変化する1度限りのパスワード。通信経路上でパスワードを盗まれた場合でも,次回のアクセスでは無効となる。実装方法には,チャレンジ アンド レスポンス方式などがある。使い捨てパスワード。OTP。''\footfullcite{スーパー大辞林}
    \end{block}
    \vfill
    今回用いるのは,\textbf{軽量}かつ\textbf{高速}のワンタイムパスワード認証方式,\textbf{\color{red} SAS-L(1)}\footnote{清水明宏教授考案}を採用する.
\end{frame}
\subsection{SAS-L(1)\ 認証フロー(概要)}
\begin{frame}{\ft}
    \begin{figure}
    \centering
    \begin{tikzpicture}[remember picture]
        %  -- TILE --
        \node[terminal](cl){Supplicant};
        \node[terminal, right=3cm of cl](sv){Server};
        \node[key={cl}{0.5}](ck1){AuthCode1};
        \node[key={sv}{0.5}](sk1){AuthCode1};
        \node[key={ck1}{3.5}](ck2){AuthCode2};
        \node[key={sk1}{3.5}](sk2){AuthCode2};
        \node[decision, below=.5cm of sk1](decision){AuthCode1 ?};
        \node[decision] at ($(ck1|-decision.east)!0.5!(ck2.north)$)(decision2){AuthCode1' ?};

        % -- ARROW --
        \draw[Stealth-Stealth,ultra thick, draw=blue!50](ck1)--(sk1)node[midway,above]{\small 初回登録};
        \begin{scope}[-Stealth, ultra thick, draw=red!50]
            \draw(ck1)|-(decision.west)node[midway, above right]{\small 認証子送信};
            \draw(decision.south)|-(decision2)node[midway, above left]{\small 認証子送信};
        \end{scope}

        \foreach \u \v in {sk1/decision,decision2.south/ck2,ck1/decision2.north,decision.south/sk2}{
                \draw[-latex](\u)--(\v);
            }
        \node[below right] at (decision.south){\scriptsize\ttfamily True};
        \node[below right] at (decision2.south){\scriptsize\ttfamily True};
    \end{tikzpicture}
\end{figure}
\begin{tikzpicture}[remember picture, overlay]
    \node[above right]at(decision.east){\ttfamily\scriptsize False};
    \draw[-latex, dashed](decision.east)-|($(decision.east)+(.5cm,-.5cm)$)node[below]{\ttfamily Failed};
    \node[above left]at(decision2.west){\ttfamily\scriptsize False};
    \draw[-latex, dashed](decision2.west)-|($(decision2.west)+(-.5cm,-.5cm)$)node[below]{\ttfamily Failed};
\end{tikzpicture}
\end{frame}
\subsection{解決すべき課題}
\begin{frame}[t]{\ft}
    \begin{alertblock}{相互認証\hfill\textbf{\fbox{\scriptsize 解決済}}}
        現行のSAS-L(1)は,
        \begin{itemize}
            \item サーバはクライアントを認証するが,
            \item クライアントはサーバを認証しない.
        \end{itemize}
        IEEE802.1X/EAP認証には,さまざまな種類がある.
        クライアントがサーバを認証しないEAP-MD5は,\textbf{クライアントが\underline{不正な}サーバにアクセスする可能性}を考える.
        認証プロトコルが相互認証機能を有する必要がある.
    \end{alertblock}
    \begin{alertblock}{同期ずれ\hfill\textbf{\fbox{\scriptsize 解決済}}}
        何らかの障害で,サーバとクライアント間の保持する認証子が同一でなくなる現象(同期ずれ)を解決する必要がある.
    \end{alertblock}
\end{frame}