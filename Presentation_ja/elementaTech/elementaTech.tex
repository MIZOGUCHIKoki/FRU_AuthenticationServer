\section{用語解説}
\begin{frame}[c]{\fft}
    \begin{block}{IEEE802.1X}
        `` 認められた機器のみがネットワークにアクセスできるように認証する仕組み''\footfullcite{マスタリングTCPIP}
    \end{block}
    \vfill
    \begin{block}{EAP}
        EAP(Extensible Authentication Protocol:直訳すると「拡張認証プロトコル」)とは,
        {{RFC3748}}及び{{RFC5247}}で規定されている,認証プロトコル.
        これに準拠することで汎用性が高まる.(後程)
    \end{block}
\end{frame}
\begin{frame}[c]{\fft}
    \begin{block}{AAAモデル\footfullcite{RADIUS}}
        \begin{itemize}
            \item {\large A}uthentication(認証)\\
                  相手は誰なのか?ネットワーク接続を許可するか?
            \item {\large A}uthorization(承認)\\
                  どのようなサービスを許可するか?\\(ネットーワークアクセス制限など)
            \item {\large A}ccounting(課金)\\
                  情報の収集.接続時間,送受信したデータ量など.
        \end{itemize}
    \end{block}
\end{frame}
\begin{frame}[c]{\fft}
    \begin{block}{RADIUS\footfullcite{RADIUS}}
        \begin{itemize}
            \item 相手が本当に名乗っているとおりユーザなのか確認.
            \item ユーザが何にアクセスできるのかの判断.
            \item そういったことすべて知らせる処理をする.
        \end{itemize}
        これらのすべてに対応したプロトコル.\par
        AAAはRADIUSプロトコルの動作を保証するための汎用的なバックグラウンド.
    \end{block}
\end{frame}