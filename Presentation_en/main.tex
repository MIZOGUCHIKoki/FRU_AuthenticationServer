\documentclass[aspectratio=43]{beamer}

% Import packages
\usepackage{luatexja}
\usepackage[symbol]{footmisc}

% Tabular setting
\usepackage{tabularx}
\newcolumntype{C}{>{\centering\arraybackslash}X}
\newcolumntype{R}{>{\raggedright\arraybackslash}X}
\newcolumntype{L}{>{\raggedleft\arraybackslash}X}

% Hyperref setting
\usepackage{hyperref}
\hypersetup{
    colorlinks=false,
}

% Margin setting
\setbeamersize{text margin left=5mm,text margin right=5mm}

% Beamer template settings
\usetheme{Montpellier}
\usecolortheme[RGB={140,0,0}]{structure}
\setbeamertemplate{navigation symbols}{}

% Beamer color box settings
\setbeamercolor{block title}{bg=black, fg=white}
\setbeamercolor{block body}{bg=black!10, fg=black}
\setbeamercolor{block title alerted}{bg=red!70!black, fg=white}
\setbeamercolor{block body alerted}{bg=red!10, fg=black}
\setbeamercolor{block title example}{bg=green!70!black, fg=white}
\setbeamercolor{block body example}{bg=green!10, fg=black}

% Font setting
\renewcommand{\familydefault}{\rmdefault}
\usefonttheme{professionalfonts}

% Caption and footnote settings
\renewcommand{\figurename}{図}
\renewcommand{\tablename}{表}
\renewcommand{\thefootnote}{\ \fnsymbol{footnote}}
\renewcommand{\thempfootnote}{\ \fnsymbol{mpfootnote}}

% Biber settings
\usepackage[backend=biber, style=numeric, sorting=none, isbn=false, url=false, doi=false,eprint=false]{biblatex}
\addbibresource{bib.bib}
\setbeamertemplate{bibliography item}[text]

% itemize & enumirate settings
\setbeamertemplate{emumerate item}[default]
\setbeamertemplate{itemize item}[triangle]

% Other settings
\makeatletter
\@addtoreset{footnote}{page}
\makeatother

% TikZ settings
\input{TikZsetting.tset}

% Footer setting
\setbeamertemplate{footline}{
    \leavevmode
    \hbox{\begin{beamercolorbox}[wd=.95\paperwidth, ht=1ex, right,dp=1.5ex, rightskip=0mm]{structure}
            \insertframenumber{} / \inserttotalframenumber
        \end{beamercolorbox}}
}

% icon settings
\usepackage{pifont}
\newcommand{\cmark}{{\Large\bfseries\textcolor{green!50!black}{\ding{51}}}}
\newcommand{\xmark}{{\Large\bfseries\textcolor{red!50!black}{\ding{55}}}}

% Table of Contents and Frame title settings
\newcommand{\stoc}{1-4}
\newcommand{\ftoc}{5-7}
\newcommand{\ft}{\thesection.\thesubsection\ \subsecname}
\newcommand{\fft}{\thesection.\ \secname}
\newcommand{\toc}{
    \begin{frame}[t]{\fft}
        \begin{columns}[t]
            \begin{column}{.48\textwidth}
                \tableofcontents[sections=\stoc, currentsection,sectionstyle=show/shaded, subsectionstyle=show/show/shaded]
            \end{column}
            \begin{column}{.48\textwidth}
                \tableofcontents[sections=\ftoc, currentsection,sectionstyle=show/shaded, subsectionstyle=show/show/shaded]
            \end{column}
        \end{columns}
    \end{frame}
}
\setbeamertemplate{section in toc}{
    {\color{structure}\inserttocsectionnumber}.\
    \inserttocsection\vspace{.7em}
}
\setbeamertemplate{subsection in toc}{
    \hspace{1em}{\color{structure}\rule[.3ex]{3pt}{3pt}}\
    \inserttocsubsection\par\vspace{.7em}
}

\title{IEEE802.1X Authentication\\ using One Time Password}
\subtitle{- Provides Simple, Secure and Fast network authentication. -}
\author[K.MIZOGUCHI]{MIZOGUCHI Koki\thanks{Kochi University of Technology, Information Security System Lab. 3rd.}}
\date{September 29th, 2023}

\begin{document}
\maketitle

% Table of Contents ---
\begin{frame}[t]{{Table of Contents}}
    \begin{columns}[t]
        \begin{column}{.48\textwidth}
            \tableofcontents[sections=\stoc]
        \end{column}
        \begin{column}{.48\textwidth}
            \tableofcontents[sections=\ftoc]
        \end{column}
    \end{columns}
\end{frame}


\section{Glossary}
\toc
\begin{frame}[c]{\fft}
    \begin{block}{IEEE802.1X}
        A mechanism to authenticate only authorized devices to access the network.\footfullcite{マスタリングTCPIP}
    \end{block}
    \vfill
    \begin{block}{EAP (Extensible Authentication Protocol)}
        EAP is a authentication protocol stipulated {{RFC3748}} and {{RFC5247}}.
        It will be more versatility to use EAP.
    \end{block}
\end{frame}
\begin{frame}[c]{\fft}
    \begin{block}{AAA model\footfullcite{RADIUS}}
        \begin{itemize}
            \item {\large A}uthentication\\
                  Who is the communication partner?
            \item {\large A}uthorization\\
                  What services are allowed?\\(Network access restrictions, etc.)
            \item {\large A}ccounting\\
                  Information collection.\\
                  (Connection time, amount of data sent/received, etc.)
        \end{itemize}
    \end{block}
\end{frame}
\begin{frame}[c]{\fft}
    \begin{block}{RADIUS\footfullcite{RADIUS}}
        RADIUS is a protocol. It is supported following items.
        \begin{itemize}
            \item Authentication for users.
            \item Determining what the user can access to.
            \item Processing to notify admin of all of those things.
        \end{itemize}
        AAA is a generic background to guaranntee the operation of the RADIUS protocol.
    \end{block}
\end{frame}
\section{IEEE802.1X認証}
\toc
\subsection{構成図}
\begin{frame}{\ft}
    \begin{figure}
    \centering
    \tikzset{base/.style={text width=6cm, minimum height=1cm, text centered, font=\ttfamily},
        supplicant/.style={base, draw, rounded corners, text width=2.8cm},
        switch/.style={base, draw, double, rectangle, font=\normalfont, fill=gray!10},
        server/.style={base, draw, thick, rectangle}
    }
    \begin{tikzpicture}[remember picture]
        \node[switch](switch){\ttfamily Swich supported IEEE802.1X};
        \node[server,above=1cm of switch](radius){RADIUS Server};
        \node[supplicant,below=1cm of switch.south west, anchor=north west](sc1){Supplicant};
        \node[supplicant,below=1cm of switch.south east, anchor=north east](sc2){Supplicant};

        \begin{scope}[red, thick, latex-latex]
            \draw(sc1.north)--(sc1.north |- switch.south)node[midway](eapol1){};
            \draw(sc2.north)--(sc2.north |- switch.south)node[midway](eapol2){};
            \node at ($(eapol1)!0.5!(eapol2)$){\ttfamily EAP over LAN};
        \end{scope}
        \begin{scope}[blue, thick, latex-latex]
            \draw(switch.north)--(radius.south)node[midway, right]{\ttfamily EAP over RADIUS};
        \end{scope}
    \end{tikzpicture}
\end{figure}
\begin{tikzpicture}[remember picture, overlay]
    \node[below right=1cm of switch.east, fill=cyan!10, rounded corners, minimum height=1cm](internet){\ttfamily Internet};
    \draw[dashed, Stealth-Stealth, thick](switch.east)-|(internet);
\end{tikzpicture}
\end{frame}
\subsection{プロトコルレイヤ}
\begin{frame}{\ft}
    \begin{figure}
    \centering
    \tikzset{box/.style n args={2}{text width=2cm, text centered, font=\ttfamily\scriptsize, minimum height=#1,fill=#2, align=center}}
    \begin{tikzpicture}
        \node[box={3cm}{green!30}](s1){IEEE802.3\\IEEE802.11};
        \node[box={1cm}{green!10},above=0cm of s1](s2){EAPOL};
        \node[box={1cm}{red!30},above=0cm of s2](s3){EAP};
        \node[box={1cm}{red!10},above=0cm of s3](s4){Application layer};

        \node[box={3cm}{green!30},right=1cm of s1](sw1){IEEE802.3\\IEEE802.11};
        \node[box={1cm}{green!10},above=0cm of sw1](sw2){EAPOL};

        \node[box={1cm}{blue!40},right=0cm of sw1.south east, anchor=south west](r1){Ethernet};
        \node[box={1cm}{blue!30},above=-0.4pt of r1](r2){IP};
        \node[box={1cm}{blue!20},above=-0.4pt of r2](r3){UDP};
        \node[box={1cm}{blue!10},above=0cm of r3](r4){RADIUS};

        \node[box={1cm}{blue!40},right=1cm of r1](rs1){Ethernet};
        \node[box={1cm}{blue!30},above=-0.4pt of rs1](rs2){IP};
        \node[box={1cm}{blue!20},above=-0.4pt of rs2](rs3){UDP};
        \node[box={1cm}{blue!10},above=0cm of rs3](rs4){RADIUS};
        \node[box={1cm}{red!30},above=0cm of rs4](rs5){EAP};
        \node[box={1cm}{red!10},above=0cm of rs5](rs6){Application layer};

        \begin{scope}[very thick, latex-latex]
            \draw(s2)--(sw2);
            \draw(s3)--(rs5);
            \draw(s4)--(rs6);
            \draw(r4)--(rs4);
        \end{scope}
        \node[below]at(s1.south){\ttfamily\scriptsize Supplicant};
        \node[below]at($(sw1.south)!0.5!(r1.south)$){\ttfamily\scriptsize Swich supported IEEE802.1X};
        \node[below]at(rs1.south){\ttfamily\scriptsize RADIUS Server};
    \end{tikzpicture}
\end{figure}
\end{frame}
\section{EAPの種類}
\toc
\begin{frame}[t]{\fft}
    \begin{itemize}
        \item IEEE802.1X認証で使用されるEAP認証方式(代表的なもの)
    \end{itemize}
    \begin{table}
        \centering
        \renewcommand{\arraystretch}{1.5}
        \begin{tabularx}{\textwidth}{cCCCC}
                                     & {\scriptsize EAP-MD5} & {\scriptsize PEAP} & {\scriptsize EAP-TTLS} & {\scriptsize EAP-TLS} \\
            \hline
            ServerEC\footnotemark[1] & 不要                    & 要                  & 要                      & 要                     \\
            ClientEC\footnotemark[1] & 不要                    & 不要                 & 不要                     & 要                     \\
            ClientAuth               & ID / PW               & ID / PW            & ID / PW                & 電子証明書                 \\
            ServerAuth               & NONE                  & 電子証明書              & 電子証明書                  & 電子証明書                 \\
            \hline
        \end{tabularx}
    \end{table}
    \footnotetext[1]{EC:Electronic Certificate}
\end{frame}
\section{ワンタイムパスワード}
\toc
\begin{frame}[c]{\fft}
    \begin{block}{ワンタイムパスワード}
        ``ネットワーク上の認証において,毎回変化する1度限りのパスワード。通信経路上でパスワードを盗まれた場合でも,次回のアクセスでは無効となる。実装方法には,チャレンジ アンド レスポンス方式などがある。使い捨てパスワード。OTP。''\footfullcite{スーパー大辞林}
    \end{block}
    \vfill
    今回用いるのは,\textbf{軽量}かつ\textbf{高速}のワンタイムパスワード認証方式,\textbf{\color{red} SAS-L(1)}\footnote{清水明宏教授考案}を採用する.
\end{frame}
\subsection{SAS-L(1)\ 認証フロー(概要)}
\begin{frame}{\ft}
    \begin{figure}
    \centering
    \begin{tikzpicture}[remember picture]
        %  -- TILE --
        \node[terminal](cl){Supplicant};
        \node[terminal, right=3cm of cl](sv){Server};
        \node[key={cl}{0.5}](ck1){AuthCode1};
        \node[key={sv}{0.5}](sk1){AuthCode1};
        \node[key={ck1}{3.5}](ck2){AuthCode2};
        \node[key={sk1}{3.5}](sk2){AuthCode2};
        \node[decision, below=.5cm of sk1](decision){AuthCode1 ?};
        \node[decision] at ($(ck1|-decision.east)!0.5!(ck2.north)$)(decision2){AuthCode1' ?};

        % -- ARROW --
        \draw[Stealth-Stealth,ultra thick, draw=blue!50](ck1)--(sk1)node[midway,above]{\small 初回登録};
        \begin{scope}[-Stealth, ultra thick, draw=red!50]
            \draw(ck1)|-(decision.west)node[midway, above right]{\small 認証子送信};
            \draw(decision.south)|-(decision2)node[midway, above left]{\small 認証子送信};
        \end{scope}

        \foreach \u \v in {sk1/decision,decision2.south/ck2,ck1/decision2.north,decision.south/sk2}{
                \draw[-latex](\u)--(\v);
            }
        \node[below right] at (decision.south){\scriptsize\ttfamily True};
        \node[below right] at (decision2.south){\scriptsize\ttfamily True};
    \end{tikzpicture}
\end{figure}
\begin{tikzpicture}[remember picture, overlay]
    \node[above right]at(decision.east){\ttfamily\scriptsize False};
    \draw[-latex, dashed](decision.east)-|($(decision.east)+(.5cm,-.5cm)$)node[below]{\ttfamily Failed};
    \node[above left]at(decision2.west){\ttfamily\scriptsize False};
    \draw[-latex, dashed](decision2.west)-|($(decision2.west)+(-.5cm,-.5cm)$)node[below]{\ttfamily Failed};
\end{tikzpicture}
\end{frame}
\subsection{解決すべき課題}
\begin{frame}[t]{\ft}
    \begin{alertblock}{相互認証\hfill\textbf{\fbox{\scriptsize 解決済}}}
        現行のSAS-L(1)は,
        \begin{itemize}
            \item サーバはクライアントを認証するが,
            \item クライアントはサーバを認証しない.
        \end{itemize}
        IEEE802.1X/EAP認証には,さまざまな種類がある.
        クライアントがサーバを認証しないEAP-MD5は,\textbf{クライアントが\underline{不正な}サーバにアクセスする可能性}を考える.
        認証プロトコルが相互認証機能を有する必要がある.
    \end{alertblock}
    \begin{alertblock}{同期ずれ\hfill\textbf{\fbox{\scriptsize 解決済}}}
        何らかの障害で,サーバとクライアント間の保持する認証子が同一でなくなる現象(同期ずれ)を解決する必要がある.
    \end{alertblock}
\end{frame}
\section{研究の目的}
\toc
\subsection{従来の課題}
\begin{frame}{\ft}
    \begin{alertblock}{}
        \begin{itemize}
            \setlength{\itemsep}{1em}
            \item \textbf{相互に電子証明書認証する方式}(EAP-TLS,\dots)\\\vspace{.5em}
                  電子証明書の導入コスト,運用コストがとにかくかかる.
            \item \textbf{ID,パスワードを用いて認証する方式}(EAP-TTLS,\dots)\\\vspace{.5em}
                  ID,パスワードが漏洩すると,認証として使い物にならない.
        \end{itemize}
    \end{alertblock}
\end{frame}
\subsection{課題解決へのアプローチ}
\begin{frame}{\ft}
    \begin{exampleblock}{}
        \begin{itemize}
            \setlength{\itemsep}{1em}
            \item \textbf{機密事項漏洩のリスクを減らす}\\
                  ID,パスワードを用いた認証を用いないことで,機密事項漏洩を防ぐ.
            \item \textbf{コスト削減}\\
                  サーバとクライアントの認証に,電子証明書を用いないことで,コスト削減を図る.
            \item \textbf{処理の高速化}\\
                  電子証明書を用いた認証ではなく,単純な認証子に対するビット演算を用いて認証することで,高速化を図る.
        \end{itemize}
    \end{exampleblock}
    ここで,ワンタイムパスワードを用いたIEEE802.1X認証,「\textit{\bfseries EAP-SAS}」を提案する.
    \textbf{\color{red}EAPに準拠して}認証規格を作成する.
\end{frame}
\section{EAP-SAS Processing}
\toc
\subsection{Protocol Layer}
\begin{frame}{\ft}
    \begin{figure}
    \centering
    \tikzset{box/.style n args={2}{text width=2cm, text centered, font=\ttfamily\scriptsize, minimum height=#1,fill=#2, align=center}}
    \begin{tikzpicture}
        \node[box={3cm}{green!30}](s1){IEEE802.3\\IEEE802.11};
        \node[box={1cm}{green!10},above=0cm of s1](s2){EAPOL};
        \node[box={1cm}{red!30},above=0cm of s2](s3){EAP};
        \node[box={1cm}{red!10},above=0cm of s3](s4){SAS-L(1)};

        \node[box={3cm}{green!30},right=1cm of s1](sw1){IEEE802.3\\IEEE802.11};
        \node[box={1cm}{green!10},above=0cm of sw1](sw2){EAPOL};

        \node[box={1cm}{blue!40},right=0cm of sw1.south east, anchor=south west](r1){Ethernet};
        \node[box={1cm}{blue!30},above=-0.4pt of r1](r2){IP};
        \node[box={1cm}{blue!20},above=-0.4pt of r2](r3){UDP};
        \node[box={1cm}{blue!10},above=0cm of r3](r4){RADIUS};

        \node[box={1cm}{blue!40},right=1cm of r1](rs1){Ethernet};
        \node[box={1cm}{blue!30},above=-0.4pt of rs1](rs2){IP};
        \node[box={1cm}{blue!20},above=-0.4pt of rs2](rs3){UDP};
        \node[box={1cm}{blue!10},above=0cm of rs3](rs4){RADIUS};
        \node[box={1cm}{red!30},above=0cm of rs4](rs5){EAP};
        \node[box={1cm}{red!10},above=0cm of rs5](rs6){SAS-L(1)};

        \begin{scope}[very thick, latex-latex]
            \draw(s2)--(sw2);
            \draw(s3)--(rs5);
            \draw(s4)--(rs6);
            \draw(r4)--(rs4);
        \end{scope}
        \node[below]at(s1.south){\ttfamily\scriptsize Supplicant};
        \node[below]at($(sw1.south)!0.5!(r1.south)$){\ttfamily\scriptsize IEEE802.1X認証対応スイッチ};
        \node[below]at(rs1.south){\ttfamily\scriptsize RADIUSサーバ};
    \end{tikzpicture}
\end{figure}
\end{frame}
\subsection{Authentication Flow}
\begin{frame}{\ft}
    \begin{figure}
    \centering
    \tikzset{tt/.style={font=\ttfamily\scriptsize,fill=white,midway,draw}}
    \begin{tikzpicture}
        \node(sp){\ttfamily Supplicant};
        \node[right=4.5cm of sp.center,anchor=center](sw){\ttfamily Switch};
        \node[right=4.5cm of sw.center,anchor=center](sv){\ttfamily RADIUS Server};
        \draw[ultra thick](sp)--($(sp)+(0,-6cm)$);
        \draw[ultra thick](sw)--($(sw)+(0,-6cm)$);
        \draw[ultra thick](sv)--($(sv)+(0,-6cm)$);

        \begin{scope}[very thick,-latex]
            \draw[blue]($(sp)-(0,.8cm)$)--($(sw)-(0,.8cm)$)node[tt]{EAPOL-Start};
            \draw[blue]($(sw)-(0,1.6cm)$)--($(sp)-(0,1.6cm)$)node[tt]{EAP-Request/Identity};

            \draw[red]($(sp)-(0,2.4cm)$)--($(sw)-(0,2.4cm)$)node[tt]{EAP-Response/Identity};
            \draw[red]($(sw)-(0,2.4cm)$)--($(sv)-(0,2.4cm)$)node[tt]{RADIUS Access-Request};

            \draw[blue]($(sv)-(0,3.2cm)$)--($(sw)-(0,3.2cm)$)node[tt]{RADIUS Access-Challenge};
            \draw[blue]($(sw)-(0,3.2cm)$)--($(sp)-(0,3.2cm)$)node[tt]{EAP-Request/Auth};

            \draw[red]($(sp)-(0,4.6cm)$)--($(sw)-(0,4.6cm)$)node[tt]{EAP-Response/Auth};
            \draw[red]($(sw)-(0,4.6cm)$)--($(sv)-(0,4.6cm)$)node[tt]{RADIUS Access-Request};

            \draw[blue]($(sv)-(0,5.4cm)$)--($(sw)-(0,5.4cm)$)node[tt]{RADIUS Access-Accept};
            \draw[blue]($(sw)-(0,5.4cm)$)--($(sp)-(0,5.4cm)$)node[tt]{EAP-Success};
        \end{scope}
        \node at ($($(sp)-(0,3.2cm)$)!0.5!($(sw)-(0,4.6cm)$)$){\large\bfseries\vdots};
        \node at ($($(sw)-(0,3.2cm)$)!0.5!($(sv)-(0,4.6cm)$)$){\large\bfseries\vdots};
    \end{tikzpicture}
\end{figure}
\end{frame}
\begin{frame}{\ft}
    \begin{figure}
    \centering
    \begin{tikzpicture}
        \node(sp){\ttfamily Supplicant};
        \node[right=6cm of sp](sv){\ttfamily Server};
        \draw[ultra thick](sp)--($(sp.south)+(0,-6cm)$);
        \draw[ultra thick](sv)--($(sv.south)+(0,-6cm)$);

        \begin{scope}[thick]
            \draw[latex-latex]($(sp.south)+(0,-.5cm)$)--($(sv.south)+(0,-.5cm)$)node[midway,fill=white]{Association};
            \draw[draw=blue,-latex]($(sp.south)+(0,-1cm)$)--($(sv.south)+(0,-1cm)$)node[midway,fill=white]{\ttfamily EAPOL Start};
            \draw[draw=red,-latex]($(sv.south)+(0,-1.5cm)$)--($(sp.south)+(0,-1.5cm)$)node[midway,fill=white]{\ttfamily EAP-Request/Identity};
            \draw[draw=blue,-latex]($(sp.south)+(0,-2cm)$)--($(sv.south)+(0,-2cm)$)node[midway,fill=white]{\ttfamily EAP-Response/Identity};
            \draw[draw=red,-latex]($(sv.south)+(0,-2.5cm)$)--($(sp.south)+(0,-2.5cm)$)node[midway,fill=white]{\ttfamily EAP-Request/Auth};
            \draw[draw=blue,-latex]($(sp.south)+(0,-3.5cm)$)--($(sv.south)+(0,-3.5cm)$)node[midway,above]{\ttfamily EAP-Response/Auth (\(\alpha, \beta\))};
            \draw[draw=red,-latex]($(sv.south)+(0,-4.3cm)$)--($(sp.south)+(0,-4.3cm)$)node[midway,above]{\ttfamily EAP-Request/Auth (\(\gamma\))};
            \draw[draw=blue,-latex]($(sp.south)+(0,-5.1cm)$)--($(sv.south)+(0,-5.1cm)$)node[midway,above]{\ttfamily EAP-Response/Auth (\(\gamma\)\quad isCorrect?)};
            \draw[draw=red,-latex]($(sv.south)+(0,-5.9cm)$)--($(sp.south)+(0,-5.9cm)$)node[midway,above]{\ttfamily EAP-Success};
        \end{scope}
    \end{tikzpicture}
\end{figure}
\end{frame}
\section{Progress and Plan}
\toc
\begin{frame}{\fft}
    \begin{alertblock}{Issues related to SAS-L(1)}
        \begin{itemize}
            \setlength{\itemsep}{1em}
            \item[\cmark] When returning the \texttt{ACK} from the client to the server,
                if the same verifire \(\gamma\) is returned, there is possibility of unauthrized access.
            \item Check to see if it can withstand a replay attack. \\
                  For any combination of verifire, I have to prove inductively that we cannnot compute the necessary verifire for authentication this time.
        \end{itemize}
    \end{alertblock}
\end{frame}
\begin{frame}{\fft}
    \begin{exampleblock}{Progress}
        \begin{itemize}
            \item[\cmark] Establish the communication sequence.
            \item[\cmark] Design packets.
            \item[\cmark] Check the authentication protocol.
            \item A Research on Cryptography protocols between the swich supported IEEE802.1X and the RADIUS server.
            \item A Research on RADIUS VSA using in PMK delivery.
            \item A Research on how supplicant can make the OS recognize PMK.
        \end{itemize}
    \end{exampleblock}
    Implementing using Python.\\
    \hfill I aim to complete it by the end of this year!
\end{frame}
\section{Reference}
\begin{frame}[allowframebreaks]{Reference}
    \printbibliography
\end{frame}
\end{document}